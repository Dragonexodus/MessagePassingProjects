%allgemeine Formatangaben
\documentclass[
 a4paper, 										% Papierformat
 12pt,												% Schriftgröße
 ngerman, 										% für Umlaute, Silbentrennung etc.
 %titlepage,										% es wird eine Titelseite verwendet
 oneside, 										% einseitiges Dokument
 captions=nooneline,					% einzeilige Gleitobjekttitel ohne Sonderbehandlung wie mehrzeilige Gleitobjekttitel behandeln
 numbers=noenddot,						% Überschriften-??Nummerierung ohne Punkt am Ende
 parskip=half,									% zwischen Absätzen wird eine halbe Zeile eingefügt
 ]{scrartcl}

% Anpassung an Landessprache
\usepackage[ngerman]{babel}	

\usepackage[T1]{fontenc}	
\usepackage[utf8]{inputenc}	
\usepackage{textcomp} 																% Euro-Zeichen und andere
\usepackage[babel,german=quotes]{csquotes}						% Anführungszeichen
\RequirePackage[ngerman=ngerman-x-latest]{hyphsubst} 	% erweiterte Silbentrennung

% Befehle aus AMSTeX für mathematische Symbole z.B. \boldsymbol \mathbb
\usepackage{amsmath,amsfonts}

% Zeilenabstände und Seitenränder 
\usepackage{setspace}
\usepackage{geometry}

% Einbinden von JPG-Grafiken
\usepackage{graphicx}

% zum Umfließen von Bildern
% Verwendung unter http://de.wikibooks.org/wiki/LaTeX-Kompendium:_Baukastensystem#textumflossene_Bilder
\usepackage{floatflt}

% Verwendung von vordefinierten Farbnamen zur Colorierung
% Palette und Verwendung unter http://kitt.cl.uzh.ch/kitt/CLinZ.CH/src/Kurse/archiv/LaTeX-Kurs-Farben.pdf
\usepackage[usenames,dvipsnames]{color} 

% Tabellen
\usepackage{array}
\usepackage{longtable}

% einfache Grafiken im Code
% Einführung unter http://www.math.uni-rostock.de/~dittmer/bsp/pstricks-bsp.pdf
\usepackage{pstricks}

% Quellcodeansichten
\usepackage{verbatim}
\usepackage{moreverb} 											% für erweiterte Optionen der verbatim Umgebung
% Befehle und Beispiele unter http://www.ctex.org/documents/packages/verbatim/moreverb.pdf
\usepackage{listings}
\lstset{literate=
  {á}{{\'a}}1 {é}{{\'e}}1 {í}{{\'i}}1 {ó}{{\'o}}1 {ú}{{\'u}}1
  {Á}{{\'A}}1 {É}{{\'E}}1 {Í}{{\'I}}1 {Ó}{{\'O}}1 {Ú}{{\'U}}1
  {à}{{\`a}}1 {è}{{\`e}}1 {ì}{{\`i}}1 {ò}{{\`o}}1 {ù}{{\`u}}1
  {À}{{\`A}}1 {È}{{\'E}}1 {Ì}{{\`I}}1 {Ò}{{\`O}}1 {Ù}{{\`U}}1
  {ä}{{\"a}}1 {ë}{{\"e}}1 {ï}{{\"i}}1 {ö}{{\"o}}1 {ü}{{\"u}}1
  {Ä}{{\"A}}1 {Ë}{{\"E}}1 {Ï}{{\"I}}1 {Ö}{{\"O}}1 {Ü}{{\"U}}1
  {â}{{\^a}}1 {ê}{{\^e}}1 {î}{{\^i}}1 {ô}{{\^o}}1 {û}{{\^u}}1
  {Â}{{\^A}}1 {Ê}{{\^E}}1 {Î}{{\^I}}1 {Ô}{{\^O}}1 {Û}{{\^U}}1
  {œ}{{\oe}}1 {Œ}{{\OE}}1 {æ}{{\ae}}1 {Æ}{{\AE}}1 {ß}{{\ss}}1
  {ű}{{\H{u}}}1 {Ű}{{\H{U}}}1 {ő}{{\H{o}}}1 {Ő}{{\H{O}}}1
  {ç}{{\c c}}1 {Ç}{{\c C}}1 {ø}{{\o}}1 {å}{{\r a}}1 {Å}{{\r A}}1
  {€}{{\euro}}1 {£}{{\pounds}}1 {«}{{\guillemotleft}}1
  {»}{{\guillemotright}}1 {ñ}{{\~n}}1 {Ñ}{{\~N}}1 {¿}{{?`}}1
} 											% für angepasste Quellcodeansichten siehe
% Kurzeinführung unter http://blog.robert-kummer.de/2006/04/latex-quellcode-listing.html

\usepackage{pgfplots}
\usepackage{pgfplotstable}
\usepackage{filecontents}
\pgfplotsset{compat=1.9}

% verlinktes und Farblich angepasstes Inhaltsverzeichnis
\usepackage[pdftex,
colorlinks=true,
linkcolor=InterneLinkfarbe,
urlcolor=ExterneLinkfarbe]{hyperref}
\usepackage[all]{hypcap}

% URL verlinken, lange URLs umbrechen
\usepackage{url}

% sorgt dafür, dass Leerzeichen hinter parameterlosen Makros nicht als Makroendezeichen interpretiert werden
\usepackage{xspace}

% Beschriftungen für Abbildungen und Tabellen
\usepackage{caption}

% Entwicklerwarnmeldungen entfernen
\usepackage{scrhack}

\newcommand{\qq}[1]{\glqq{#1\grqq{}}} %Gänsefüßchen

\onehalfspacing 							% 1,5facher Zeilenabstand

\definecolor{InterneLinkfarbe}{rgb}{0.1,0.1,0.3} 	% Farbliche Absetzung von externen Links
\definecolor{ExterneLinkfarbe}{rgb}{0.1,0.1,0.7}	% Farbliche Absetzung von internen Links

% Einstellungen für Fußnoten:
\captionsetup{font=footnotesize,labelfont=sc,singlelinecheck=true,margin={5mm,5mm}}	
					
\title{Dokumentation zu den Projektaufgaben 1 und 2}
\subtitle{Message Passing Programmierung}

\author{Matthias Zober und Michael Horn}
\date{\today}
\begin{document}
\maketitle

\tableofcontents
\pagebreak

\section{Einleitung}
In dieser Dokumentation werden die Aufgaben \qq{Rechteckmustererkennung} sowie \qq{Numerische Integration mittels Parabelformel} betrachtet und hinsichtlich ihrer Parallelität untersucht.
Der entstandene Quellcode kann unter:\\
\url{https://github.com/MZober1993/MessagePassingProjects}\footnote{Letzter Aufruf: \today}
eingesehen werden.
\section{Projektaufgabe 1: Rechteckmustererkennung}

\subsection{Realisierung}
%Pseudocode? Verwendung (config), argumente,...
%Einschränkung (gerade Prozessoranzahl etc)
%Vll. Scatter und Gather nochmal erklären


%Speed-Up , Effizienz
%Wo liegt das Maximum für das zu untersuchende quadratische Raster?
%Kommunikationsoverhead
\subsection{Laufzeitverhalten}
% Auswertungsoverhead durch Master (Auswertung von Gather - parallel)

\subsubsection{Festes n}
\begin{tikzpicture}
%TODO csv wird noch nicht richtig gelesen, falsche formatierung?
%Tests mit manuell angepasster csv haben geklappt... o.O
\begin{axis}
[            axis x line=middle,
            axis y line=middle,
            enlarge y limits=true,
            width=15cm, height=8cm,     % size of the image
            grid = major,
            grid style={dashed, gray!30},
            ylabel=Time in s,
            xlabel=n,
            legend style={at={(0.1,-0.1)}, anchor=north}
]
\addplot[black, mark=x] table [x=n, y=T2 , col sep=space] {../rectangle/measures/nScaling/88/measure0.csv};
\addplot[blue, mark=*] table [x=n, y=T2 , col sep=space] {../rectangle/measures/nScaling/88/measure1.csv};
\addplot[red, mark=+] table [x=n, y=T2 , col sep=space] {../rectangle/measures/nScaling/88/measure2.csv};
\legend{0,1,2}
\end{axis}
\end{tikzpicture}

\begin{tikzpicture}
%TODO csv wird noch nicht richtig gelesen, falsche formatierung?
%Tests mit manuell angepasster csv haben geklappt... o.O
\begin{axis}
[            axis x line=middle,
            axis y line=middle,
            enlarge y limits=true,
            width=15cm, height=8cm,     % size of the image
            grid = major,
            grid style={dashed, gray!30},
            ylabel=Time in s,
            xlabel=n,
            legend style={at={(0.1,-0.1)}, anchor=north}
]
\addplot[black, mark=x] table [x=n, y=T2 , col sep=space] {../rectangle/measures/nScaling/44/measure0.csv};
\addplot[blue, mark=*] table [x=n, y=T2 , col sep=space] {../rectangle/measures/nScaling/44/measure1.csv};
\addplot[red, mark=+] table [x=n, y=T2 , col sep=space] {../rectangle/measures/nScaling/44/measure2.csv};
\legend{0,1,2}
\end{axis}
\end{tikzpicture}
\subsubsection{Festes p}
\begin{tikzpicture}
%TODO csv wird noch nicht richtig gelesen, falsche formatierung?
%Tests mit manuell angepasster csv haben geklappt... o.O
\begin{axis}
[            axis x line=middle,
            axis y line=middle,
            enlarge y limits=true,
            width=15cm, height=8cm,     % size of the image
            grid = major,
            grid style={dashed, gray!30},
            ylabel=Time in s,
            xlabel=hosts,
            legend style={at={(0.1,-0.1)}, anchor=north}
]
\addplot[black, mark=x] table [x=hosts, y=T1 , col sep=space] {../rectangle/measures/processorScaling/44/40000/measure0.csv};
\addplot[blue, mark=*] table [x=hosts, y=T1 , col sep=space] {../rectangle/measures/processorScaling/44/40000/measure1.csv};
\addplot[red, mark=+] table [x=hosts, y=T1 , col sep=space] {../rectangle/measures/processorScaling/44/40000/measure2.csv};
\legend{0,1,2}
\end{axis}
\end{tikzpicture}

\begin{tikzpicture}
%TODO csv wird noch nicht richtig gelesen, falsche formatierung?
%Tests mit manuell angepasster csv haben geklappt... o.O
\begin{axis}
[            axis x line=middle,
            axis y line=middle,
            enlarge y limits=true,
            width=15cm, height=8cm,     % size of the image
            grid = major,
            grid style={dashed, gray!30},
            ylabel=Time in s,
            xlabel=hosts,
            legend style={at={(0.1,-0.1)}, anchor=north}
]
\addplot[black, mark=x] table [x=hosts, y=T2 , col sep=space] {../rectangle/measures/processorScaling/88/40000/measure0.csv};
\addplot[blue, mark=*] table [x=hosts, y=T2 , col sep=space] {../rectangle/measures/processorScaling/88/40000/measure1.csv};
\addplot[red, mark=+] table [x=hosts, y=T2 , col sep=space] {../rectangle/measures/processorScaling/88/40000/measure2.csv};
\legend{0,1,2}
\end{axis}
\end{tikzpicture}
\subsection{Fazit}

\pagebreak

\section{Projektaufgabe 2: Numerische Integration mittels Parabelformel}
\lstset{language=Fortran,frame=none, keepspaces=false, basicstyle=\footnotesize,showstringspaces=false}  

\subsection{Verwendung}
Der Quellcode kann mit Hilfe folgendes Aufrufes auf dem MC-3-System kompiliert werden:
\begin{itemize}
	\item f77.px -o A2star.px A2star.f
\end{itemize}
Mit folgendem Befehl wird das Programm ausgeführt:
\begin{itemize}
	\item run -f0 4 2 A2star.px \textbf{funktion} \textbf{n}
\end{itemize}

Für das Programm können Kommandozeilenargumente zur Auswahl der Testfunktion und die Größe der zu berechnenden Teilstücke des Integrals angegeben werden.
\paragraph{funktion:}
Das Programmargument \textbf{funktion} entscheidet darüber, welche der beiden gegebenen Testfunktionen verwendet wird.
Als gültige Eingabe wird 1 für Funktion 1 und 2 für Funktion 2 erwartet.
\paragraph{n:}
Das zweite Programmargument \textbf{n} gibt die 2er Potenz der zu berechnenden Teilstücke des Integrals an.
Je größer das \textbf{n}, umso genauer wird das berechnete PI.
De höchstmöglichste zulässige Eingabe ist 20, das sind 1048576 zu berechnende Teilstücke.
Diese Vorgehensweise soll zum Einen sicher stellen, dass \textbf{n} stets durch 2 Teilbar ist und zum Anderen das Laufzeitmessen vereinfachen, da ein exponentielles Wachstum vorliegt.
Sollte kein Programmargument angeben werden, so wird intern die Funktion 1 mit $n = 2^{15} = 32768$ verwendet.

Weiterhin wird eine ungerade Prozessorzahl ausgeschlossen, damit stets gewährleistet ist, dass die zu berechnende Anzahl an Teilstücke pro Prozessor eine ganzzahlige Zahl ist.

\subsection{Realisierung}

Grundlegend wurde sich bei der Realisierung für eine Stern-Topologie entschieden.
Gewählt wurde diese Topologie, da bei der Parallelisierung dieser Aufgabe ein Master existert, welche alle Teilergebnisse empfängt und summiert.
Weiterhin wurde diese Topologie bereits in einem Seminar implementiert.
Wie die Topologie erstellt wird zeigt \autoref{ref:top}.

\begin{lstlisting}[caption=Erzeugen der Stern-Topologie\label{ref:top}]
	if(id.eq.0) then
		do i=1,(np-1)
			links(i)=addnewlink(topid,i,1)
		enddo
	else
		link=addnewlink(topid,0,1)
	endif
\end{lstlisting}

Da beim Starten des Programms alle Prozessoren \textbf{n} sowie die Testfunktion bereits kennen, aufgrund der Realisierung als Kommandozeilenparameter, entfällt der Kommunikationsoverhead für das Verteilen von \textbf{n} sowie das Übermitteln der Information, welche Funktion verwendet wird.

Die Prozessoren können explizit ihre eigenen Bereich für die Teilstücke bestimmen und somit die numerische Berechnung durchführen.
Nach Abschluss der Berechnung ihres Ergebnisses, werden diese an den Master-Prozessor gesendet.
Der Master-Prozessor addiert die Ergebnisse der anderen Prozessoren zu seinem eigenem Ergebnis auf und wendet die Multiplikation h/3 auf das Gesamtergebnis an.
Im Anschluss werden die Ergebnisse wie Referenzwert-PI, Berechnetes-PI, Abweichung und Laufzeit (nicht im Listng dargestellt) ausgegeben. 
Das folgende \autoref{ref:main} zeigt dies.

\begin{lstlisting}[caption=Empfangen und Auswerten\label{ref:main}]
c	starte die Integration
	call startIntegration(f,n,integral,h)
	if(id.eq.0) then
		print*,"use f:",f," with n:",n
c  	Integralteilstück vom Master
		summe = integral
		do i=1,(np-1)
c 	Empfange alle Teilstücke und addiere Sie
			call recv(topid,links(i),integral,8)
			summe = summe + integral
		enddo
		summe = h/3 *summe
		derivation =  PI() - summe
		print*,"--------------------------"
		print*,"PI Referece:",PI()                   
		print*,"PI Integral:",summe
		print*,"PI Derivat.:",derivation
	else
c	Sende Integralteilstücke zum Master
		call send(topid, link,integral,8)
	endif
\end{lstlisting}

Die numerische Berechnung des Integrals wird durch jeden Prozessor ausgeführt, aber jeder Prozessor führt nur einen Teil der kompletten Berechnung aus.
Die eigentliche Berechnung ist in dem \autoref{ref:intervall} zu sehen.
Zunächst berechnet jeder Prozessor die Anzahl der durchzuführenden Berechnungen, als lokales \textbf{n} (\textbf{nL}).
Im Anschluss werden Startpunkt (\textbf{aL}) und Endpunkt (\textbf{bL}) ermittelt.
Die Funktionswerte von \textbf{aL} und \textbf{bL} der Testfunktion werden addiert und in einer Summe gespeichert.
Im Anschluss werden die Zwischenstücke mit entsprechendem Faktor berechnet und ebenfalls summiert.
Die Variable \textbf{h} gibt dabei die allgemeine Schrittweite an und \textbf{step} die derzeitigen Schrittwert.

Nach erfolgter Berechnung sendet jeder Prozessor sein Ergebnis an den Master (falls der Prozessor nicht selbst der Master ist)

\begin{lstlisting}[caption=Berechnen der relevanten Bereiche für jeden Prozessor\label{ref:intervall}]
	id = myprocid() 
	nL = n / nprocs()
	aL = a + id * nL * h
	bL = aL + nL * h
c	Randwerte, getVal liefert Funktionswert der Testfunktion
	summe = getVal(f,aL) + getVal(f,bL)   
	step = aL + h
c 	Berechnung zwischenstücke, factor alterniert mit 2 und 4
	do i=1,nL-1
		exponent = mod(i,2) + 1
		factor=2**exponent
		summe = summe + dble(getVal(f,step) * factor)
		step = step + h    
	enddo
\end{lstlisting}

\subsection{Ergebnisse}
%TODO Tabellen

\subsubsection{Testfunktionen}

Wie in den Tabellen zu sehen ist liefert die Testfunktion 2 schneller bessere PI-Werte als Funktion 1.
Ein Grund dafür das Testfunktion 1 schlechtere Ergebnisse als Funktion 2 liefert, könnte damit zusammenhängen, dass für die Funktion 1 die Oberegrenze PI selbst ist.
%TODO Unterschiede Genauigkeit der Funktionen mit Begründung

%Speed-Up , Effizienz
\subsubsection{Laufzeitverhalten}
\paragraph{Festes n}
\paragraph{Festes p}

\subsection{Fazit}
Kommunikationsoverhead spielt hier eine untergeordnete Rolle, da Minimum an Kommunikation zur Berechnung notwendig ist.
%TODO
Weiterhin lohnt sich die Parallelisierung,..


%\appendix
%\section{Quellcode Aufgabe 1}
%
%\section{Quellcode Aufgabe 2}
%\lstset{language=Fortran,numbers=left, keepspaces=false, basicstyle=\footnotesize,showstringspaces=false} 
%\lstinputlisting[]{../integration/A2star.f}

\end{document}
