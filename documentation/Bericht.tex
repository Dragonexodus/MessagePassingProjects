%allgemeine Formatangaben
\documentclass[
 a4paper, 										% Papierformat
 12pt,												% Schriftgröße
 ngerman, 										% für Umlaute, Silbentrennung etc.
 %titlepage,										% es wird eine Titelseite verwendet
 oneside, 										% einseitiges Dokument
 captions=nooneline,					% einzeilige Gleitobjekttitel ohne Sonderbehandlung wie mehrzeilige Gleitobjekttitel behandeln
 numbers=noenddot,						% Überschriften-??Nummerierung ohne Punkt am Ende
 parskip=half,									% zwischen Absätzen wird eine halbe Zeile eingefügt
 ]{scrartcl}

% Anpassung an Landessprache
\usepackage[ngerman]{babel}	

\usepackage[T1]{fontenc}	
\usepackage[utf8]{inputenc}	
\usepackage{textcomp} 																% Euro-Zeichen und andere
\usepackage[babel,german=quotes]{csquotes}						% Anführungszeichen
\RequirePackage[ngerman=ngerman-x-latest]{hyphsubst} 	% erweiterte Silbentrennung

% Befehle aus AMSTeX für mathematische Symbole z.B. \boldsymbol \mathbb
\usepackage{amsmath,amsfonts}

% Zeilenabstände und Seitenränder 
\usepackage{setspace}
\usepackage{geometry}

% Einbinden von JPG-Grafiken
\usepackage{graphicx}

% zum Umfließen von Bildern
% Verwendung unter http://de.wikibooks.org/wiki/LaTeX-Kompendium:_Baukastensystem#textumflossene_Bilder
\usepackage{floatflt}

% Verwendung von vordefinierten Farbnamen zur Colorierung
% Palette und Verwendung unter http://kitt.cl.uzh.ch/kitt/CLinZ.CH/src/Kurse/archiv/LaTeX-Kurs-Farben.pdf
\usepackage[usenames,dvipsnames]{color} 

% Tabellen
\usepackage{array}
\usepackage{longtable}

% einfache Grafiken im Code
% Einführung unter http://www.math.uni-rostock.de/~dittmer/bsp/pstricks-bsp.pdf
\usepackage{pstricks}

% Quellcodeansichten
\usepackage{verbatim}
\usepackage{moreverb} 											% für erweiterte Optionen der verbatim Umgebung
% Befehle und Beispiele unter http://www.ctex.org/documents/packages/verbatim/moreverb.pdf
\usepackage{listings}
\lstset{literate=
  {á}{{\'a}}1 {é}{{\'e}}1 {í}{{\'i}}1 {ó}{{\'o}}1 {ú}{{\'u}}1
  {Á}{{\'A}}1 {É}{{\'E}}1 {Í}{{\'I}}1 {Ó}{{\'O}}1 {Ú}{{\'U}}1
  {à}{{\`a}}1 {è}{{\`e}}1 {ì}{{\`i}}1 {ò}{{\`o}}1 {ù}{{\`u}}1
  {À}{{\`A}}1 {È}{{\'E}}1 {Ì}{{\`I}}1 {Ò}{{\`O}}1 {Ù}{{\`U}}1
  {ä}{{\"a}}1 {ë}{{\"e}}1 {ï}{{\"i}}1 {ö}{{\"o}}1 {ü}{{\"u}}1
  {Ä}{{\"A}}1 {Ë}{{\"E}}1 {Ï}{{\"I}}1 {Ö}{{\"O}}1 {Ü}{{\"U}}1
  {â}{{\^a}}1 {ê}{{\^e}}1 {î}{{\^i}}1 {ô}{{\^o}}1 {û}{{\^u}}1
  {Â}{{\^A}}1 {Ê}{{\^E}}1 {Î}{{\^I}}1 {Ô}{{\^O}}1 {Û}{{\^U}}1
  {œ}{{\oe}}1 {Œ}{{\OE}}1 {æ}{{\ae}}1 {Æ}{{\AE}}1 {ß}{{\ss}}1
  {ű}{{\H{u}}}1 {Ű}{{\H{U}}}1 {ő}{{\H{o}}}1 {Ő}{{\H{O}}}1
  {ç}{{\c c}}1 {Ç}{{\c C}}1 {ø}{{\o}}1 {å}{{\r a}}1 {Å}{{\r A}}1
  {€}{{\euro}}1 {£}{{\pounds}}1 {«}{{\guillemotleft}}1
  {»}{{\guillemotright}}1 {ñ}{{\~n}}1 {Ñ}{{\~N}}1 {¿}{{?`}}1
} 											% für angepasste Quellcodeansichten siehe
% Kurzeinführung unter http://blog.robert-kummer.de/2006/04/latex-quellcode-listing.html

\usepackage{pgfplots}
\usepackage{pgfplotstable}
\usepackage{filecontents}
\pgfplotsset{compat=1.9}

% verlinktes und Farblich angepasstes Inhaltsverzeichnis
\usepackage[pdftex,
colorlinks=true,
linkcolor=InterneLinkfarbe,
urlcolor=ExterneLinkfarbe]{hyperref}
\usepackage[all]{hypcap}

% URL verlinken, lange URLs umbrechen
\usepackage{url}

% sorgt dafür, dass Leerzeichen hinter parameterlosen Makros nicht als Makroendezeichen interpretiert werden
\usepackage{xspace}

% Beschriftungen für Abbildungen und Tabellen
\usepackage{caption}

% Entwicklerwarnmeldungen entfernen
\usepackage{scrhack}

\newcommand{\qq}[1]{\glqq{#1\grqq{}}} %Gänsefüßchen

\onehalfspacing 							% 1,5facher Zeilenabstand

\definecolor{InterneLinkfarbe}{rgb}{0.1,0.1,0.3} 	% Farbliche Absetzung von externen Links
\definecolor{ExterneLinkfarbe}{rgb}{0.1,0.1,0.7}	% Farbliche Absetzung von internen Links

% Einstellungen für Fußnoten:
\captionsetup{font=footnotesize,labelfont=sc,singlelinecheck=true,margin={5mm,5mm}}	

\usepackage{pgfplots}

\usepackage{filecontents}
					
\title{Dokumentation zu den Projektaufgaben 1 und 2}
\subtitle{Message Passing Programmierung\vspace{1cm}}

\author{M. Zober, M. Horn}
\date{\today}
\begin{document}
\maketitle

\tableofcontents
\pagebreak

\section{Einleitung}
In dieser Dokumentation werden die Aufgaben \qq{Rechteckmustererkennung} sowie \qq{Numerische Integration mittels Parabelformel} betrachtet und hinsichtlich ihrer Parallelität untersucht.
Der entstandene Quellcode kann unter:\\
\url{https://github.com/MZober1993/MessagePassingProjects}\footnote{Letzter Aufruf: \today}
eingesehen werden.
\section{Projektaufgabe 1: Rechteckmustererkennung}

\subsection{Realisierung}
%Pseudocode? Verwendung (config), argumente,...
%Einschränkung (gerade Prozessoranzahl etc)
%Vll. Scatter und Gather nochmal erklären


%Speed-Up , Effizienz
%Wo liegt das Maximum für das zu untersuchende quadratische Raster?
%Kommunikationsoverhead
\subsection{Laufzeitverhalten}
% Auswertungsoverhead durch Master (Auswertung von Gather - parallel)
%Beispiel .csv
%Anzahl	T-Wert
%4812	7.027466627
%8989	9.4135177029
%12032	10.0600658745
%94399	10.0270125326
%200580	10.0075516954

\begin{tikzpicture}
%TODO csv wird noch nicht richtig gelesen, falsche formatierung?
%Tests mit manuell angepasster csv haben geklappt... o.O
\begin{axis}
[            axis x line=middle,
            axis y line=middle,
            enlarge y limits=true,
            width=15cm, height=8cm,     % size of the image
            grid = major,
            grid style={dashed, gray!30},
            ylabel=Time in s,
            xlabel=hosts,
            legend style={at={(0.1,-0.1)}, anchor=north}
]
\addplot[black, mark=x] table [x=hosts, y=T1 , col sep=space] {../rectangle/measures/processorScaling/44/10000/measure.csv};
\legend{Messwerte}
\end{axis}
\end{tikzpicture}

\subsubsection{Festes n}
\subsubsection{Festes p}

\subsection{Fazit}

\section{Projektaufgabe 2: Numerische Integration mittels Parabelformel}
%Pseudocode? Verwendung (config), argumente,...
%Einschränkung (gerade Prozessoranzahl etc)
%Vll. Scatter und Gather nochmal erklären
Als Topologie für diese Projekt wurde sich dafür entschieden die Stern-Topologie zu verwenden.


% Topologie, 
% Unterschiede Genauigkeit der Funktionen mit Begründung
% Kommunikationsoverhead?
\subsection{Realisierung}
% Geringer Kommunikationsoverhead
% Nur Recv., da jeder Prozessor sich seinen teilbereich selbst ausrechnen kann


%Speed-Up , Effizienz
\subsection{Laufzeitverhalten}
\subsubsection{Festes n}
\subsection{Festes p}

\subsection{Fazit}


\end{document}
