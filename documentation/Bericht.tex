\input{Header}	

\usepackage{pgfplots}

\usepackage{filecontents}
					
\title{Dokumentation zu den Projektaufgaben 1 und 2}
\subtitle{Message Passing Programmierung\vspace{1cm}}

\author{M. Zober, M. Horn}
\date{\today}
\begin{document}
\maketitle

\tableofcontents
\pagebreak

\section{Einleitung}
In dieser Dokumentation werden die Aufgaben \qq{Rechteckmustererkennung} sowie \qq{Numerische Integration mittels Parabelformel} betrachtet und hinsichtlich ihrer Parallelität untersucht.
Der entstandene Quellcode kann unter:\\
\url{https://github.com/MZober1993/MessagePassingProjects}\footnote{Letzter Aufruf: \today}
eingesehen werden.
\section{Projektaufgabe 1: Rechteckmustererkennung}

\subsection{Realisierung}
%Pseudocode? Verwendung (config), argumente,...
%Einschränkung (gerade Prozessoranzahl etc)
%Vll. Scatter und Gather nochmal erklären


%Speed-Up , Effizienz
%Wo liegt das Maximum für das zu untersuchende quadratische Raster?
%Kommunikationsoverhead
\subsection{Laufzeitverhalten}
% Auswertungsoverhead durch Master (Auswertung von Gather - parallel)

\subsubsection{Festes n}
\begin{tikzpicture}
%TODO csv wird noch nicht richtig gelesen, falsche formatierung?
%Tests mit manuell angepasster csv haben geklappt... o.O
\begin{axis}
[            axis x line=middle,
            axis y line=middle,
            enlarge y limits=true,
            width=15cm, height=8cm,     % size of the image
            grid = major,
            grid style={dashed, gray!30},
            ylabel=Time in s,
            xlabel=n,
            legend style={at={(0.1,-0.1)}, anchor=north}
]
\addplot[black, mark=x] table [x=n, y=T2 , col sep=space] {../rectangle/measures/nScaling/88/measure0.csv};
\addplot[blue, mark=*] table [x=n, y=T2 , col sep=space] {../rectangle/measures/nScaling/88/measure1.csv};
\addplot[red, mark=+] table [x=n, y=T2 , col sep=space] {../rectangle/measures/nScaling/88/measure2.csv};
\legend{0,1,2}
\end{axis}
\end{tikzpicture}

\begin{tikzpicture}
%TODO csv wird noch nicht richtig gelesen, falsche formatierung?
%Tests mit manuell angepasster csv haben geklappt... o.O
\begin{axis}
[            axis x line=middle,
            axis y line=middle,
            enlarge y limits=true,
            width=15cm, height=8cm,     % size of the image
            grid = major,
            grid style={dashed, gray!30},
            ylabel=Time in s,
            xlabel=n,
            legend style={at={(0.1,-0.1)}, anchor=north}
]
\addplot[black, mark=x] table [x=n, y=T2 , col sep=space] {../rectangle/measures/nScaling/44/measure0.csv};
\addplot[blue, mark=*] table [x=n, y=T2 , col sep=space] {../rectangle/measures/nScaling/44/measure1.csv};
\addplot[red, mark=+] table [x=n, y=T2 , col sep=space] {../rectangle/measures/nScaling/44/measure2.csv};
\legend{0,1,2}
\end{axis}
\end{tikzpicture}
\subsubsection{Festes p}
\begin{tikzpicture}
%TODO csv wird noch nicht richtig gelesen, falsche formatierung?
%Tests mit manuell angepasster csv haben geklappt... o.O
\begin{axis}
[            axis x line=middle,
            axis y line=middle,
            enlarge y limits=true,
            width=15cm, height=8cm,     % size of the image
            grid = major,
            grid style={dashed, gray!30},
            ylabel=Time in s,
            xlabel=hosts,
            legend style={at={(0.1,-0.1)}, anchor=north}
]
\addplot[black, mark=x] table [x=hosts, y=T1 , col sep=space] {../rectangle/measures/processorScaling/44/40000/measure0.csv};
\addplot[blue, mark=*] table [x=hosts, y=T1 , col sep=space] {../rectangle/measures/processorScaling/44/40000/measure1.csv};
\addplot[red, mark=+] table [x=hosts, y=T1 , col sep=space] {../rectangle/measures/processorScaling/44/40000/measure2.csv};
\legend{0,1,2}
\end{axis}
\end{tikzpicture}

\begin{tikzpicture}
%TODO csv wird noch nicht richtig gelesen, falsche formatierung?
%Tests mit manuell angepasster csv haben geklappt... o.O
\begin{axis}
[            axis x line=middle,
            axis y line=middle,
            enlarge y limits=true,
            width=15cm, height=8cm,     % size of the image
            grid = major,
            grid style={dashed, gray!30},
            ylabel=Time in s,
            xlabel=hosts,
            legend style={at={(0.1,-0.1)}, anchor=north}
]
\addplot[black, mark=x] table [x=hosts, y=T2 , col sep=space] {../rectangle/measures/processorScaling/88/40000/measure0.csv};
\addplot[blue, mark=*] table [x=hosts, y=T2 , col sep=space] {../rectangle/measures/processorScaling/88/40000/measure1.csv};
\addplot[red, mark=+] table [x=hosts, y=T2 , col sep=space] {../rectangle/measures/processorScaling/88/40000/measure2.csv};
\legend{0,1,2}
\end{axis}
\end{tikzpicture}
\subsection{Fazit}

\section{Projektaufgabe 2: Numerische Integration mittels Parabelformel}
%Pseudocode? Verwendung (config), argumente,...
%Einschränkung (gerade Prozessoranzahl etc)
%Vll. Scatter und Gather nochmal erklären
Als Topologie für diese Projekt wurde sich dafür entschieden die Stern-Topologie zu verwenden.


% Topologie, 
% Unterschiede Genauigkeit der Funktionen mit Begründung
% Kommunikationsoverhead?
\subsection{Realisierung}
% Geringer Kommunikationsoverhead
% Nur Recv., da jeder Prozessor sich seinen teilbereich selbst ausrechnen kann


%Speed-Up , Effizienz
\subsection{Laufzeitverhalten}
\subsubsection{Festes n}
\subsection{Festes p}

\subsection{Fazit}


\end{document}
